\section{Realios mašinos projektas}

\subsection{Techninės įrangos komponentai}

\begin{description}
  \item[Procesorius] 
  \item[Vartotojo atmintis] atmintis skirta virtualių mašinų atmintims bei 
    puslapių lentelėms laikyti. Jos dydis 256 blokai po 16 žodžių, kur
    žodžio ilgis yra 8 baitai. Atmintis numeruojama nuo 0. 

    Pirmieji 16 blokų yra rezervuoti puslapiavimo mechanizmui. Registru
    \verb|PLR| nurodoma, kuriame bloke yra saugoma lentelė, o registru
    \verb|PLBR| nurodoma, kuriuo baitu bloke prasideda puslapių lentelė.
    Atminties baite, kurio pozicija yra $PLR \cdot 16 + PLBR$ ir po jo 
    esančiame baite yra saugoma, kiek blokų užima kodo segmentas 
    (pažymėkime $C (1 \leq C)$). Kitoje baitų poroje nurodyta kiek blokų 
    užima duomenų segmentas (pažymėkime $D (0 \leq D)$). Toliau yra 
    $C$ baitų porų, kuriose nurodyta kokie blokai atitinka kodo puslapius.
    Analogiškai, po jų yra $D$ baitų porų, kurie nurodo kokie blokai 
    atitinka, kuriuos duomenų puslapius.

    Kiti 240 blokų yra skirti virtualių mašinų duomenims.
  \item[Supervizorinė atmintis] jos dydis 16 blokų.
  \item[Išorinė atmintis] kietasis diskas.
  \item[Duomenų perdavimo kanalai] yra 4. 
    \begin{description}
      \item[1 kanalu] perduodami duomenys iš įvedimo įrenginio į 
        supervizorinę atmintį.
      \item[2 kanalu] perduodami duomenys iš supervizorinės atminties
        į išvedimo įrenginį.
      \item[3 kanalu] perduodami duomenys iš kietojo disko į supervizorinę
        atmintį.
      \item[4 kanalu] perduodami duomenys iš supervizorinės atminties į 
        kietąjį diską.
      \item[5 kanalu] perduodami duomenys iš supervizorinės atminties į 
        vartotojo atmintį.
    \end{description}
  \item[Įvedimo ir išvedimo įrenginiai] klaviatūra ir ekranas.
\end{description}

Komandų argumentai nurodyti 16-ainiais skaičiais.

Programuotojas turi nurodyti kokio dydžio duomenų segmentą 
išskirti programai.
