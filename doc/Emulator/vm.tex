\section{Virtualios mašinos aprašas}

\subsection{Virtualios mašinos samprata}

Virtuali mašina – tai tarsi realios mašinos kopija. Virtualioje mašinoje
surenkame mums reikalingus komponentus, tokius kaip procesorius,
atmintis, įvedimo/išvedimo įrenginiai, suteikiame jiems paprastesnę
naudotojo sąsają. Tuo pačiu palengvinamas programavimo procesas,
nes sudėtingas ar naudotojui nepatogias sąsajas virtualioje mašinoje yra 
aprašomos supaprastintai. Virtuali mašina realizuoja realios mašinos
komandas paprastesniu, lengviau suprantamu būdu interpretuojant virtualios
mašinos komandas kaip realios mašinos komandas ar jų rinkiniu. Taip pat 
virtuali mašina pateikia supaprastintą atminties adresavimą. Visa tai 
leidžia pasiekti realią mašiną ir virtualios mašinos mašininiu kodu 
parašytą programą sėkmingai įvykdyti realioje mašinoje. 

\subsection{Virtualios mašinos komponentų aprašymas}

\begin{description}
  \item[Atmintis] virtualiai mašinai yra paskiriama jos inicijavimo metu,
    o jos dydis nurodytas programos metainformacijoje.
    Ji yra padalinta į dvi dalis – vykdomąjį kodą ir duomenis. Virtuali
    mašina gali keisti tik duomenų dalyje esančią informaciją, o vykdyti
    tik kodo dalyje esančias komandas. Virtuali mašina dirba su 
    8 baitų ilgio žodžiais. Tiek kodo tiek duomenų dalių adresai 
    skaičiuojami nuo 0.
  \item[Procesorius] turi registrus:
    \begin{description}
      \item[IC] Nuorodos. 3 baitų dydžio, skirtas nurodyti vykdomos komandos
        adresą virtualioje atmintyje.
      \item[R1] Duomenų. 8 baitų bendro naudojimo registras.
      \item[R2] Duomenų. 8 baitų bendro naudojimo registras.
      \item[SF] Loginis. 2 baitų registras, skirtas saugoti aritmetinių 
        operacijų loginių (teisinga „1“ arba klaidinga „0“) reikšmių sekai. 
        Lentelėje pateikta informacija apie logines reikšmes:

        \begin{tabularx}{0.85\textwidth}{|c|c|c|X|} 
          \hline
          Bitas & Trumpinys & Reikšmė & Paaiškinimas \\
          \hline
          0 & CF & pernešimo požymis & įgija „1“ tada, kai sudėties arba
          atimties rezultatas netelpa į žodį \\
          \hline
          6 & ZF & nulio požymis & parodo ar paskutinės operacijos 
          rezultatas yra nulinis \\
          \hline
          7 & SF & ženklo požymis & parodo, koks yra paskutinės operacijos 
          rezultato ženklas („1“ – jei neigiamas) \\
          \hline
          11 & OF & perpildymo požymis & įgija „1“ tada, kai rezultatas
          netelpa skaičių su ženklu diapazone \\
          \hline
        \end{tabularx}
    \end{description}
  \item[Komandų sistema] \hfill
    \begin{description} 
        \item[$LR1 \: x$] Atminties ląstelės, kurios adresas 
          $x (x $ – triženklis šešioliktainis skaičius), 
          turinio kopijavimas į registrą $R1 (R1:=[x])$.
        \item[$LR2 \: x$] Atminties ląstelės, kurios adresas 
          $x$, turinio kopijavimas į registrą $R2 (R2:=[x])$.
        \item[$SR1 \: x$] Registro $R1$ turinio kopijavimas į atminties 
          ląstelę, kurios adresas $x ([x]:=R1)$.
        \item[$SR2 \: x$] Registro $R2$ turinio kopijavimas į atminties 
          ląstelę, kurios adresas $x ([x]:=R2)$.
        \item[$ADD$] Prie registro $R1$ reikšmės pridedama registro $R2$ 
          reikšmė. Formuoja $SF$ požymius. $(R1:=R1+R2)$.
        \item[$ADDM \: x$] Sudedamos registrų $R1$ ir $R2$ reikšmės bei 
          rezultatas išsaugojamas atminties ląstelėje, kurios adresas 
          $x$. Formuoja $SF$ požymius. $([x]:=R1+R2)$.
        \item[$SUB$] Iš registro $R1$ reikšmės atimama registro $R1$ 
          reikšmė. Formuoja $SF$ požymius. $(R1:=R1-R2)$.
        \item[$SUBM \: x$] Iš registro $R1$ atimama registro $R2$ 
          reikšmė ir rezultatas įrašomas į atminties ląstelę, kurios 
          adresas $x$. Formuoja $SF$ požymius. $([x]:=R1-R2)$.
        \item[$DIV$] Registro $R1$ reikšmė padalinama iš registro $R2$ 
          reikšmės ir dalmuo įrašomas į registrą $R1$, o liekana - į 
          registrą $R2$. Formuoja $SF$ požymius.
        \item[$MUL$] Registro $R1$ reikšmė padauginama iš registro $R2$ 
          reikšmės ir rezultatas įrašomas į registrą $R1$. 
          Formuoja $SF$ požymius.
        \item[$CMP$] Palygina registrus $R1$ ir $R2$. Formuoja $SF$ požymius
          ($R1>R2$, tai $ZF=0$ ir $SF=0; R1=R2$, tai $ZF=1; R1<R2$, 
          tai $ZF=0$ ir $SF=1$).
        \item[$JMP \: x$] Nesąlyginis valdymo perdavimas kodo segmento 
          komandai adresu  $x (IC:=x)$.
        \item[$JE \: x$] Jei $ZF=1$, tai valdymas perduodamas kodo segmento 
          komandai adresu $x$.
        \item[$JA \: x$] Jei $ZF=0$ ir $SF=0$, tai valdymas perduodamas kodo
          segmento komandai adresu $x$.
        \item[$JNB \: x$] Jei $SF=0$, tai valdymas perduodamas 
          kodo segmento komandai adresu $x$.
        \item[$JB \: x$] Jei $SF=1$, tai valdymas perduodamas kodo segmento
          komandai adresu $x$.
        \item[$JNA \: x$] Jei $ZF = 1 \lor SF=1$, tai valdymas perduodamas 
          kodo segmento komandai adresu $x$.
        \item[$JC \: x$] Jei $CF = 1$, tai valdymas perduodamas kodo 
          segmento komandai adresu $x$.
        \item[$JO \: x$] Jei $OF = 1$, tai valdymas perduodamas kodo 
          segmento komandai adresu $x$.
        \item[$PD \: x \: y$] Išsiunčia bloką (pradedant adresu 
          $y$) į failo abstrakciją, kurios $id = x$.
        \item[$PDR \: y$] Išsiunčia bloką (pradedant adresu 
          $y$) į failo abstrakciją, kurios $id$ yra nurodytas $R1$.
        \item[$GD \: x \: y$] Nuskaito bloką iš failo abstrakcijos, kurios
          $id = x$ į atmintį pradedant žodžiu, kurio adresas yra $y$.
        \item[$GDR \: y$] Nuskaito bloką iš failo abstrakcijos, kurios
          $id$ yra nurodytas $R1$, į atmintį pradedant žodžiu, kurio 
          adresas yra $y$.
        \item[$FO \: x \: y$] Atidaro failą, kurio pavadinimas yra saugomas
          žodyje, kurio adresas yra $y$, rėžimu 
          $x (x \in \left\{ r, w \right\})$. Atidaryto failo $id$ 
          išsaugo į $R1$.
        \item[$FC x$] Uždaromas failas, kurio $id$ yra $x$.
        \item[$FCR$] Uždaromas failas, kurio $id$ yra $R1$.
        \item[$FD x$] Pašalinti failą, kurio pavadinimas saugomas atmintyje
          adresu $x$.
        \item[$HALT$] Užbaigti programos darbą.
    \end{description}
  \end{description} 
  
\subsection{Virtualios mašinos bendravimo su įvedimo/išvedimo įrenginiais
mechanizmo aprašymas}

Virtualiai mašinai įvedimo ir išvedimo įrenginiai atrodo lygiai taip pat,
kaip paprasti failai, esantys išorinėje atmintyje. Kiekvienas atidarytas
failas turi savo numerį $id (0 \leq id < 4)$, kuris yra nurodomas 
instrukcijose. Paleidžiant virtualią mašiną yra automatiškai atidaromi
du specialieji failai: vienas skaitymui (su $id = 0$), kuris iš tiesų yra
įvedimo įrenginys ir vienas rašymui (su $id = 1$), kuris iš tiesų yra 
išvedimo įrenginys. Ne specialiųjų failų $id$ virtuali mašina sužino
atidarydama failą su komanda 
$FO\,\text{ <rėžimas> <failo pavadinimo adresas>}$,
kuri atidaryto failo $id$ įrašo į registrą $R1$.

  %XXX Jei procesas veikia supervizoriaus rėžimu, tai jis yra VM ar nėra?
  % Tai yra ar egzistuoja tokia VM, kuri gali tiesiogiai bendrauti su
  % įvedimo ir išvedimo įrenginiais?

\subsection{Virtualios mašinos interpretuojamojo vykdomojo failo išeities 
teksto formatas}

Kiekvienas vykdomasis failas turi prasidėti eilute
\verb|#!/usr/bin/pyemu|.

Toliau turi būti eilutė su raktiniu žodžiu \verb|CODE|, kuris nurodo
kodo bloko pradžią. Toliau turi būti pateikiamas programos komandos – 
kiekviena atskiroje eilutėje. Prieš komandą gali būti nurodyta žymė
(sintaksė: \verb|<žymė>: <komanda>|), kurią paskui galima naudoti vykdymo
perdavimo komandose vietoj betarpiško argumento (sintaksė: 
\verb|<komanda> «<žymė>»|). Kodo blokas užbaigiamas su raktiniu žodžiu
\verb|ENDCODE|

Toliau turi būti eilutė su raktiniu žodžiu \verb|DATA| ir po tarpo 
sveikuoju skaičiumi – duomenų segmento dydis blokais.
Toliau esančiose eilutėse pateikiamas duomenų segmento turinys. Naujos
eilutės simboliai yra ignoruojami. Jei eilutė prasideda 
\verb|[(?P<adresas>)]:| tai toliau esantys simboliai rašomi į duomenų
segmentą nuo nurodyto adreso. Duomenų segmentas užbaigiamas su 
raktiniu žodžiu \verb|ENDDATA|.

\subsection{Modeliuojamos virtualios mašinos loginių komponentų sąryšio su 
realios mašinos techninės įrangos komponentais aprašymas}

Virtuali mašina savo komandų vykdymui naudoja realios mašinos procesorių.
Vienintelis skirtumas yra tai, kad jai neleidžiama pasiekti, kai kurių
jai nereikalingų registrų. Taip pat virtuali mašina mato tik jai skirtą
visos naudotojo atminties dalį bei gali pasiekti tik tuos failus, kurie
yra jai paskirti.
   
%Visi skaičiai yra su ženklu!

%Komandų argumentai nurodyti 16-ainiais skaičiais.
