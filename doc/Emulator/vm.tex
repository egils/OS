\section{Virtualios mašinos aprašas}

\subsection{Virtualios mašinos samprata}

Virtuali mašina - tai tarsi realios mašinos kopija. Virtualioje mašinoje
surenkame mums reikalingus komponentus, tokius kaip procesorius,
atmintis, įvedimo/išvedimo įrenginiai, suteikiame jiems paprastesnę
vartotojo sąsają. Tuo pačiu palengvinamas programavimo procesas,
nes sudėtingas ar vartotojui nepatogias sąsajas virtualioje mašinoje yra 
aprašomos supaprastintai. Virtuali mašina realizuoja realios mašinos
komandas paprastesniu, lengviau suprantamu būdu interpretuojant virtualios
mašinos komandas kaip realios mašinos komandas ar jų rinkiniu. Taip pat 
virtuali mašina pateikia supaprastintą atminties adresavimą. Visa tai 
leidžia pasiekti realią mašiną ir virtualios mašinos mašininiu kodu 
parašytą programą sėkmingai įvykdyti realioje mašinoje. 

\subsection{Virtualios mašinos komponentų aprašymas}

\begin{description}
  \item[Atmintis] virtualiai mašinai yra paskiriama jos inicijavimo metu,
    o jos dydis nurodytas programos metainformacijoje.
    Ji yra padalinta į dvi dalis – vykdomąjį kodą ir duomenis. Virtuali
    mašina gali keisti tik duomenų dalyje esančią informaciją, o vykdyti
    tik kodo dalyje esančias komandas. Virtuali mašina dirba su 
    8 baitų ilgio žodžiais. Tiek kodo tiek duomenų dalių adresai 
    skaičiuojami nuo 0.
  \item[Procesorius] turi registrus:
    \begin{description}
      \item[IC] Nuorodos. 3 baitų dydžio, skirtas nurodyti vykdomos komandos
        adresą virtualioje atmintyje.
      \item[R1] Duomenų. 8 baitų bendro naudojimo registras.
      \item[R2] Duomenų. 8 baitų bendro naudojimo registras.
      \item[SF] Loginis. 2 baitų registras, skirtas saugoti aritmetinių operacijų 
        loginių (teisinga „1“ arba klaidinga „0“) reikšmių sekai. Lentelėje
        pateikta informacija apie logines reikšmes, bei jas suformuojanti
        \verb|C++| funkcija, kai yra sudedami du skaičiai \verb|sk1| ir
        \verb|sk2|:

        
        \begin{tabularx}{0.85\textwidth}{|c|c|c|X|}
        \hline
           Bitas & Trumpinys & Reikšmė & Paaiškinimas %& C++ funkcija 
           \\
           \hline
           0 & CF & pernešimo požymis & įgija „1“ tada, kai sudėties arba
           atimties rezultatas netelpa į žodį 
           %& \verb|\left( bool) \left( sk1+sk2) & (1<<8)))| 
           \\
           %\hline
           %2 & PF & lyginumo požymis & & 
           %\verb|!\left( bitset<8>(sk1+sk2).count(&1)| \\
           %\hline
           %4 & AF & papildomo pernešimo požymis & 
           %  \verb|\left( bool) \left( (sk1&0xff)| 
           %  \verb|+ (sk2&0xff))&(1<<4)))| \\
           \hline
          6 & ZF & nulio požymis & parodo ar paskutinės operacijos 
          rezultatas yra nulinis %& \verb|(!(sk1+sk2))| 
          \\
          \hline
          7 & SF & ženklo požymis & parodo, koks yra paskutinės operacijos 
          rezultato ženklas („1“ – jei neigiamas) 
          %& \verb|\left( bool) \left( sk1+sk2) & (1<<7)))| 
          \\
          %\hline
          %8 & TF & „spąstų“ požymis & jei „1“, tai po kiekvienos komandos
          %įvyksta pertraukimas & \\
          %9 & IF & & \\
          %10 & DF & & \\
          \hline
          11 & OF & perpildymo požymis & įgija „1“ tada, kai rezultatas
          netelpa skaičių su ženklu diapazone 
          %& \verb|\left( (sk1&(1<<7))&=& (sk2&(1<<7)))&&|
          %  \verb|\left( sk1&(1<<7))&=& \left( sk1+sk2)&(1<<7))))| 
          \\
          \hline
        \end{tabularx}
    \end{description}
  \item[Komandų sistema] 
    \begin{description}
        \item[LR1 xyz] Atminties ląstelės, kurios adresas (x*10+y)*10+z,
          turinio kopijavimas į registrą R1 (R1:=[(x*10+y)*10+z]).
        \item[LR2 xyz] Atminties ląstelės, kurios adresas (x*10+y)*10+z, 
          turinio kopijavimas į registrą R2 (R2:=[(x*10+y)*10+z]).
        \item[SR1 xyz] Registro R1 turinio kopijavimas į atminties ląstelę, 
          kurios adresas (x*10+y)*10+z ([(x*10+y)*10+z]:=R1).
        \item[SR2 xyz] Registro R2 turinio kopijavimas į atminties ląstelę,
          kurios adresas (x*10+y)*10+z ([(x*10+y)*10+z]:=R2).
        \item[ADD] Prie registro R1 reikšmės pridedama registro R2 reikšmė.
          Formuoja SF požymius. (R1:=R1+R2).
        \item[ADD xyz] Sudedamos registrų R1 ir R2 reikšmės bei rezultatas 
          išsaugojamas atminties ląstelėje, kurios adresas (x*10+y)*10+z.
          Formuoja SF požymius. ([(x*10+y)*10+z]:=R1+R2).
        \item[SUB] Iš registro R1 reikšmės atimama registro R1 reikšmė. 
          Formuoja SF požymius.
          (R1:=R1-R2).
        \item[SUB xyz] Iš registro R1 atimama registro R2 reikšmė ir 
          rezultatas įrašomas į atminties ląstelę, kurios adresas 
          (x*10+y)*10+z. Formuoja SF požymius. ([(x*10+y)*10+z]:=R1-R2).
        \item[DIV] Registro R1 reikšmė padalinama iš registro R2 reikšmės ir
          dalmuo įrašomas į registrą R1, o liekana - į registrą R2. Formuoja
          SF požymius.
          %FIXME: reikia trečio akumuliatoriaus registro arba steko, kad 
          %realizuoti dalybą!
        \item[CMP] Palygina registrus R1 ir R2. Formuoja SF požymius
          (R1>R2, tai ZF=0 ir SF=0; R1=R2, tai ZF=1; R1<R2, tai ZF=0 
          ir SF=1).
        \item[JMP] Nesąlyginis valdymo perdavimas kodo segmento žodžiui
          adresu (10*x+y)*10+z (IC:=(10*x+y)*10+z).
        \item[JE xyz] Jei ZF=0, tai valdymas perduodamas kodo segmento 
          žodžiui adresu (10*x+y)*10+z.
        \item[JG xyz] Jei ZF=0 ir SF=0, tai valdymas perduodamas kodo
          segmento žodžiui adresu (10*x+y)*10+z.
        \item[JB xyz] Jei SF=1, tai valdymas perduodamas kodo segmento
          žodžiui adresu (10*x+y)*10+z.
        \item[PD xyz] Išsiunčia 2 žodžius (pradedant adresu (10*x+y)*10+z)
          į išvesties įrenginį.
        \item[GD xyz] Į 2 žodžius (pradedant adresu (10*x+y)*10+z) 
          įrašoma informacija gauta iš įvesties įrenginio.
        \item[FO xyz] Atidaromas failas \verb|(FIXME)|%FIXME:
        \item[FGD xyz] Skaityti iš failo žodį ir jo reikšmė įrašyti į
          atminties ląstelę adresu (10*x+y)*10+z.
        \item[FPD xyz] Skaityti iš failo vieną žodį ir įrašyti jo 
          reikšmę į ląstelę, esančią adresu (10*x+y)*10+z.
        \item[FC xyz] Uždaromas failas \verb|(FIXME)|%FIXME:
        \item[FD xyz] Pašalinti failą \verb|(FIXME)|%FIXME:
        \item[HALT] Užbaigti programos darbą.`
    \end{description}
  \end{description} 
   %TODO:
   %2.3) Virtualios mašinos bendravimo su įvedimo/išvedimo įrenginiais
   %mechanizmo aprašymas.
\subsection{Virtualios mašinos bendravimo su įvedimo/išvedimo įrenginiais}

   %TODO:
   %2.4) Virtualios mašinos interpretuojamojo ar kompiliuojamo vykdomojo
   %failo išeities teksto formatas. Pavyzdžiui, kaip išskiriamas duomenų
   %segmentas, kodo segmentas, kaip aprašomi duomenys ir t.t.)
\subsection{Virtualios mašinos interpretuojamojo ar kompiliuojamo vykdomojo 
failo išeities teksto formatas}

   %TODO:
   %2.5) Modeliuojamos virtualios mašinos loginių komponentų sąryšio su
   %realios mašinos techninės įrangos komponentais aprašymas.
\subsection{Modeliuojamos virtualios mašinos loginių komponentų sąryšio su 
realios mašinos techninės įrangos komponentais aprašymas}
   
%Visi skaičiai yra su ženklu!

%Komandų argumentai nurodyti 16-ainiais skaičiais.

%Programuotojas turi nurodyti kokio dydžio duomenų segmentą 
%išskirti programai.


